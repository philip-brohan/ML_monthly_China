\documentclass[a4paper,12pt]{article} 
\usepackage[margin=2.5cm]{geometry}
\usepackage{graphicx}
\usepackage{times}
\usepackage{caption}
\usepackage{hyperref}

\title{China station locations and reanalysis quality}

\author{Philip Brohan} 
\date{1st May 2023}

\begin{document} 
\maketitle

This document is an addendum to \href{https://metoffice.sharepoint.com/:b:/r/sites/csspchinaext/MO%20Outputs%20Upload/WP1/Papers/D1.6.3_Brohan_ML_for_climate_modelling.pdf?csf=1&web=1&e=G4ypno}{{\it ML for Climate Modelling} (Brohan 2023)}. That paper describes a machine learning model architecture and an example model for UK surface weather. This document shows the use of a similar model, trained on China-region fields from ERA5, and demonstrates how to use it to look at station coverage effects.

For details of the China-region model, see the \href{https://github.com/philip-brohan/ML_monthly_China}{source code repository}. It's not quite as good as the UK version, but it represents regional climate well (figures \ref{Validation} and \ref{Validation_multi}).

\begin{figure}[h]
\center{\includegraphics[width=8.3cm]{../../ML_models/DCVAE_ERA5/validation/comparison.png}}
\caption{Validation plot for China-region model for a test month (January 1969). Left column: Target fields (from ERA5). Middle column: (model output). Right column model::target scatter. For four variables (from top): MSLP, Precipitation, T2m, and SST.}
\label{Validation}
\end{figure}

\begin{figure}[h]
\center{\includegraphics[width=8.3cm]{../../ML_models/DCVAE_ERA5/validation/multi.png}}
\caption{Time-series of regional mean for all test months (and scatter plots). Black lines show target data, red lines model output.}
\label{Validation_multi}
\end{figure}
    




\end{document}

