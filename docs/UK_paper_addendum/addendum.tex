\documentclass[a4paper,12pt]{article} 
\usepackage[margin=2.5cm]{geometry}
\usepackage{graphicx}
\usepackage{times}
\usepackage{caption}
\usepackage{subcaption}
\usepackage{hyperref}

\title{China station locations and reanalysis quality}

\author{Philip Brohan} 
\date{1st May 2023}

\begin{document} 
\maketitle

This document is an addendum to \href{https://metoffice.sharepoint.com/:b:/r/sites/csspchinaext/MO%20Outputs%20Upload/WP1/Papers/D1.6.3_Brohan_ML_for_climate_modelling.pdf?csf=1&web=1&e=G4ypno}{{\it ML for Climate Modelling} (Brohan 2023)}. That paper describes a machine learning model architecture and an example model for UK surface weather. This document shows the use of a similar model, trained on China-region fields from ERA5, and demonstrates how to use it to look at station coverage effects.

For details of the China-region model, see the \href{https://github.com/philip-brohan/ML_monthly_China}{source code repository}. It's not quite as good as the UK version, but it represents regional climate well (figure \ref{Validation}).

To use this system to look at station coverage effects, we use the same data assimilation system described in Brohan 2023, but decimate the coverage of MSLP, 2m temperature and precipitation to a set of points corresponding to station locations (we keep SST as a field). This will reproduce the effects of assimilating data only from the station locations.

Figure \ref{All_05} reproduces the validation plot, but only using data from station locations - assuming a good coverage of stations (5 degree characteristic separation).

Figure \ref{All_20} is the same, but uses many fewer stations (20 degree characteristic separation).

And figure \ref{SST+PRMSL_05} is more representative of current dynamical reanalyses - it assimilates PRMSL on;ly, at the station locations.

The system behaves as expected - with lots of stations results are good, reducing the station coverage reduces the precision of the reconstruction (but useful results are still obtained with a small number of stations), and even assimilating only PRMSL (and the SST field) there is still skill in the model output of 2m temperature and precipitation.

\begin{figure}[h]
\begin{subfigure}{1.0\textwidth}    
\center{\includegraphics[width=8.3cm]{../../ML_models/DCVAE_ERA5/validation/comparison.png}}
\caption{A test month (January 1969). Left column: Target fields (from ERA5), and station locations. Middle column: model output. Right column model::target scatter. For four variables (from top): MSLP, Precipitation, T2m, and SST.}
\end{subfigure}
\begin{subfigure}{1.0\textwidth}    
\center{\includegraphics[width=8.3cm]{../../ML_models/DCVAE_ERA5/validation/multi.png}}
\caption{Time-series of regional mean for all test months. Black lines show target data, red lines model output.}
\end{subfigure}
\caption{Validation plot for China-region model}.
\label{Validation}
\end{figure}
     
\pagebreak
\begin{figure}[h]
\begin{subfigure}{1.0\textwidth}    
\center{\includegraphics[width=8.3cm]{../../ML_models/DCVAE_ERA5/fit_to_fields/fit_points_all_05.png}}
\caption{A test month (January 1969). Left column: Target fields (from ERA5), and station locations. Middle column: model output. Right column model::target scatter. For four variables (from top): MSLP, Precipitation, T2m, and SST.}
\end{subfigure}
\begin{subfigure}{1.0\textwidth}    
\center{\includegraphics[width=8.3cm]{../../ML_models/DCVAE_ERA5/fit_to_fields/multi_points_all_05.png}}
\caption{Time-series of regional mean for all test months (and scatter plots). Black lines show target data, red lines model output.}
\end{subfigure}
\caption{Effect of assimilating all variables, with a high station density}.
\label{All_05}
\end{figure}
    
     
\pagebreak
\begin{figure}[h]
\begin{subfigure}{1.0\textwidth}    
\center{\includegraphics[width=8.3cm]{../../ML_models/DCVAE_ERA5/fit_to_fields/fit_points_all_20.png}}
\caption{A test month (January 1969). Left column: Target fields (from ERA5), and station locations. Middle column: model output. Right column model::target scatter. For four variables (from top): MSLP, Precipitation, T2m, and SST.}
\end{subfigure}
\begin{subfigure}{1.0\textwidth}    
\center{\includegraphics[width=8.3cm]{../../ML_models/DCVAE_ERA5/fit_to_fields/multi_points_all_20.png}}
\caption{Time-series of regional mean for all test months (and scatter plots). Black lines show target data, red lines model output.}
\end{subfigure}
\caption{Effect of assimilating all variables, with a low station density}.
\label{All_20}
\end{figure}

    
\pagebreak
\begin{figure}[h]
\begin{subfigure}{1.0\textwidth}    
\center{\includegraphics[width=8.3cm]{../../ML_models/DCVAE_ERA5/fit_to_fields/fit_points_sst+prmsl_05.png}}
\caption{A test month (January 1969). Left column: Target fields (from ERA5), and station locations. Middle column: model output. Right column model::target scatter. For four variables (from top): MSLP, Precipitation, T2m, and SST.}
\end{subfigure}
\begin{subfigure}{1.0\textwidth}    
\center{\includegraphics[width=8.3cm]{../../ML_models/DCVAE_ERA5/fit_to_fields/multi_points_sst+prmsl_05.png}}
\caption{Time-series of regional mean for all test months (and scatter plots). Black lines show target data, red lines model output.}
\end{subfigure}
\caption{Effect of assimilating SST and PRMSL only, with a high station density}.
\label{SST+PRMSL_05}
\end{figure}



\end{document}

